Lo scopo di questo progetto è di programmare una FPGA mediante il linguaggio VHDL, al fine di \textbf{equalizzare} delle immagini date in ingresso. Queste ultime sono memorizzate in una memoria ram di tipo sincrono.\\
Le immagini sono in \textbf{scala di grigi} e ogni pixel assume valori da 0 a 255, dove 0 corrisponde al colore nero e 255 al colore bianco. La dimensione massima di ciascuna immagine è di 128 pixel in larghezza e 128 pixel in altezza, per un peso totale di 16KB.\\
Il modulo progettato, si aspetta di trovare in memoria nei primi due indirizzi la dimensione rispettivamente di larghezza e altezza, seguite da tutti i pixel dell'immagine, riga per riga. Data l'immagine in ingresso, il componente effettua alcune semplici operazioni e restituisce l'immagine aumentandone il contrasto. La tecnica che viene utilizzata è l'\textbf{equalizzazione dell'istogramma}, che consiste nello spalmare i valori di intensità dell'immagine su tutto l'intervallo  (da 0 a 255).\\
Di seguito è riportato un esempio di immagine che è stata realmente\footnote{Per vedere in azione il modulo sintetizzato, è stato creato uno script in python che carica un'immagine nella ram del modulo FPGA e, una volta processata, renderizza l'immagine equalizzata presente in ram per poter osservare il risultato.} equalizzata dal modulo sintetizzato.

\begin{figure}[h!]
  \centering
  \begin{subfigure}[b]{0.4\linewidth}
    \includegraphics[width=\linewidth]{pre}
    \caption{Immagine da equalizzare.}
  \end{subfigure}
  \begin{subfigure}[b]{0.4\linewidth}
    \includegraphics[width=\linewidth]{post}
    \caption{Immagine equalizzata.}
  \end{subfigure}
  \caption{Esempio di equalizzazione di un'immagine.}
  \label{fig:equalized}
\end{figure}
